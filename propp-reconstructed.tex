\documentclass{article}
\usepackage{ijcai18}
\usepackage{times}

\title{Propp Reconstructed: Building Story Components From Formal Models of Tropes}

% REPLACE WITH ANON PAPER ID
\author{
Matt Thompson$^1$, 
Julian Padget$^2$, 
Steve Battle$^3$, 
\\ 
$^1$ Centre for Creative Computing, Bath Spa University, UK \\
$^2$ Dept. of Computer Science, University of Bath, UK \\
$^3$ Dept. Computer Science and Creative Technologies, University of the West of
England, Bristol, UK \\
%
m.thompson3@bathspa.ac.uk,
j.a.padget@bath.ac.uk,
steve.battle@uwe.ac.uk
% \setlength\titlebox{4.5in}
}


\begin{document}

\maketitle

\begin{abstract}
  Abstract to go here.
\end{abstract}

\section{Introduction}

\section{Related Work}

\section{Propp's Morphology of the Folktale}
% Opinions are divided about the utility of Propp
% Need a base of evidence to identify the shortcomings of both Character
% functions and story moves, in order to establish our rationale (to be
% explained in the next section) for why tropes
% are a good alternative

\section{Tropes}
% explain what tropes are and how they break stories up in different
% and better ways than Propp: present them as a refactoring (?!) of
% Propp which allows hierarchical, branching and cyclic structure
% and re-use

\section{TropICAL: A Programming Language for Tropes}
% Express what we want TropICAL to do in order to author the tropes

\section{Compiling TropICAL Tropes to Social Institutions}
% the technical stuff: will need rapid intro to event+state models

\section{Using Tropes in a Multi-Agent System}
% present agents as actors whose behaviour is guided by the story
% non-compliance with the story is either handled by the author
% (forseen violation) or by the super-trope (unforeseen violation)
% leading to action by the story manager

\section{Evaluation and Discussion}

\section{Conclusions and Future Work}


\bibliographystyle{named}
\bibliography{propp-reconstructed}

\end{document}
